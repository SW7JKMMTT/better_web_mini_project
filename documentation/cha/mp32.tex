\chapter{Miniproject 3.2 - REST}
We decided to use our todo app to showcase REST

The todo app needs two different ways to serve data, and one to add data. We
created a \code{get} endpoint on \code{/todo/api/todo}, seen in
\cref{lst:getAll} to get all the todo notes. To add new todos, we created
a \code{post} endpoint on \code{/todo/api/todo}, show in \cref{lst:postOne}. The
website posts JSON encoded objects when it wants to add a todo note. We also
wanted the ability to show a single todo note. For that we create another
\code{get} endpoint on \code{/todo/api/todo/<key>}, which can be seen in
\cref{lst:getOne}.

Please note that the example does not completely follow REST, as it only allows
responses in JSON.

\begin{listing}
    \begin{js}
router.get("/api/todo", function(req, res, next) {
	console.log("Getting")
	storage.getItem("items").then(function(items) {
		res.json(items)
	});
});
    \end{js}
    \caption{Todo get all method}\label{lst:getAll}
\end{listing}

\begin{listing}
    \begin{js}
router.post("/api/todo", function(req, res, next) {
	console.log(req.body);
	console.log("Adding " + req.body.title);
	storage.getItem("items").then(function(items) {
		var item = new Item(req.body.key, req.body.title, false)
		items.push(item);
		storage.setItem("items", items);
		res.json(item)
	});
});
    \end{js}
    \caption{Todo post one method}\label{lst:postOne}
\end{listing}

\begin{listing}
    \begin{js}
router.get("/api/todo/:t_id", function(req, res, next) {
	console.log("Getting")
	storage.getItem("items").then(function(items) {
		res.json(items[req.params.t_id])
	});
});
    \end{js}
    \caption{Todo get one method}\label{lst:getOne}
\end{listing}
