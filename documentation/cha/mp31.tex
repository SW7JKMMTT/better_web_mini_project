%!TEX root = ../main.tex
\chapter{Miniproject 3.1}
To create a SOAP server, we use a node package called ``soap-server''.
The method together with the service which is provided by the API, can be seen in \myref{lst:soap_func}.
The method only returns the concatenation of two input strings.

\begin{listing}
    \begin{js}
        function MyTestService(){
        }
        MyTestService.prototype.test1 = function(myArg1, myArg2){
            return myArg1 + myArg2;
        };
    \end{js}
    \caption{A SOAP API function.}
    \label{lst:soap_func}
\end{listing}

A request to the API can be seen in \myref{lst:soap_req}.
The first line is a curl request and the body of the request can be seen on the following lines. 
The two arguments, which is sent is ``Soapy '' and ``Hands''.

\begin{listing}
    \begin{xmlblock}
        curl --header "content-type: text/xml" -d @request.xml http://localhost:1337/testService
        <SOAP-ENV:Envelope
          xmlns:SOAP-ENV="http://schemas.xmlsoap.org/soap/envelope/"
          SOAP-ENV:encodingStyle="http://schemas.xmlsoap.org/soap/encoding/">
          <SOAP-ENV:Body>
            <m:test1 xmlns:m="http://127.0.0.1:1337/testService">
              <myArg1>Soapy </myArg1>
              <myArg2>Hands</myArg2>
            </m:test1>
          </SOAP-ENV:Body>
        </SOAP-ENV:Envelope>
    \end{xmlblock}
    \caption{A SOAP request.}
    \label{lst:soap_req}
\end{listing}

The reply from the server can be seen in \myref{lst:soap_respond}.
The reply is, as exptected ``Soapy Hands''.

\begin{listing}
    \begin{xmlblock}
        <soap:Envelope
                xmlns:soap="http://www.w3.org/2001/12/soap-envelope"
                soap:encodingStyle="http://www.w3.org/2001/12/soap-encoding">
                <soap:Body>
                        <test1Response>
                                <return>Soapy Hands</return>
                        </test1Response>
                </soap:Body>
        </soap:Envelope>
    \end{xmlblock}
    \caption{A SOAP response.}
    \label{lst:soap_respond}
\end{listing}