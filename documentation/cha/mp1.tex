%!TEX root = ../main.tex
\chapter{Miniproject 1}
We chose to use a framework called Express and React. 
It works on top of Node.js, which is a JavaScript rumtime. 

Express handles the web server and the routing to files, and React handles the rendering of the user interface.

An example of Express routing can be seen on \myref{lst:express_route} and \myref{lst:express_route_use}.

\begin{listing}
    \begin{js}
        var index = require('./routes/index');
    \end{js}
    \caption{Express GET method snippet.}
    \label{lst:express_route}
\end{listing}

On \myref{lst:express_route_use} is requests on ``/'' redirected to ``index'', which is defined in \myref{lst:express_route}.
We have made the folder structure of our project so that the logic for ``index'' is in ``./routes/index.js''.

\begin{listing}
    \begin{js}
        app.use('/', index);
    \end{js}
    \caption{Express GET method snippet.}
    \label{lst:express_route_use}
\end{listing}

An example of a GET request handle can be seen on \myref{lst:express_get}.
Express handle request with call-back functions.
``router.get'' is called when ``/api'' is requested and the function parameter is called. 
In this example it writes ``Getting'' in the console and gets a collection of items called ``items''.

\begin{listing}
    \begin{js}
        router.get("/api", function(req, res, next) {
            console.log("Getting")
            storage.getItem("items").then(function(items) {
                res.json(items)
            });
        });
    \end{js}
    \caption{Express GET method snippet.}
    \label{lst:express_get}
\end{listing}

POST requests is handled the same way as GET requests. 
In this case it adds a new item to the ``items'' collection.

\begin{listing}
    \begin{js}
        router.post("/api", function(req, res, next) {
            console.log(req.body);
            console.log("Adding " + req.body.title);
            storage.getItem("items").then(function(items) {
                var item = new Item(req.body.key, req.body.title, false)
                items.push(item);
                storage.setItem("items", items);
                res.json(item)
            });
        });
    \end{js}
    \caption{Express POST method snippet.}
    \label{lst:express_post}
\end{listing}

To render the UI we use React.
React executes on the client side and sends the POST and GET requests to the server, and the requests is processed by Express.

We use Node.js with Express since it's easy to start with and follows the MVC architecture, where the model is apart from the view and controller.
React is used to make it easy and simple to manage and update the UI. 