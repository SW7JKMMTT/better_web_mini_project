\chapter{Miniproject Part 4}
For this part of the miniproject we used the Google Analytics demo.
The problem with such tools is that you need a decent dataset before the information actually becomes usable, as such it may not always be an option for less used web applications.

\section{Quality Assessment}
Using the tool for quality assessment allows us to evaluate four quality metrics.
\begin{description}
	\item [Availability] Through the analytics tool it is possible to check site traffic, as such for the demo it is clear that since the site is always used it is always available. 
	\item [Interoperability] The analytics tool also allows you to view what type of  OS, browser and device type are used.
	While not directly being able to determine whether something works on each device, if you have 100\% of users on a specific device bouncing off the site quickly, it is reasonable to determine that your site is not appealing on that device.
	\item [Performance] Through the analytics tool we can see our site speed and number of request allowing us to evaluate performance, as well as seeing at what number of users the site speed is noticable affected.
	\item [Reliability] The tool also allows us to look closer into sessions, that way we an see if some sessions end unconventionally, which could be caused by an error. The site speed as well as site events and searches may also be used to evaluate reliability.
\end{description}

In order to gage the remaining criteria, in particular usability and functionality, other information sources are needed.

\section{Web Usage Analysis}
The analytics tool reveals a significant amount of information about site usage, particularly about sessions.
The tool allows us to see the amount of sessions as well as relevant data for each session, but also allows us to see the amount of new users, this way the analytics tool effectively allows us easy sessionization.
This of cause cannot take into account such things as VPN, proxy or dynamic IP address usage, but still allows a good estimate.
The analytics tool also allows us to see time it took to make a purchase, if one was made, as well as bounce rate.
These two factors allow us to gage at the success of the site, in particular the pages that are bounced from.
With this data available we decided to perform some session analysis.
With the data about bounce rate of almost 50 \% as well as the average time spent making a purchase is 10 minutes, with 30 \% of sessions making a purchase, we arrived at the conclusion that the initial page is either not informative enough or the users are arriving from a misleading link.
The low amount of time spent on making a purchase also indicates that most user purchases are premeditated and not impulsive.