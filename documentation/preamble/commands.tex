\newcommand{\taskhead}[1]{%
	This section describes the work on task #1.\xspace
}

\newcommand{\taskheadcont}[1]{%
  This section describes the continued work on task #1.\xspace
}

\newcommand{\taskfoot}[1]{%
	During the work we created #1.\xspace
}

\newcommand{\giraf}[1]{\texttt{#1}\xspace}
\newcommand{\android}[1]{\texttt{#1}\xspace}

\newcommand{\method}[1]{%
	\textbf{#1}\xspace
}

\newcommand{\name}[1]{%
	\textit{#1}\xspace
}

\newcommand{\rendpoint}[1]{\texttt{#1}\xspace}

\newcommand{\hsp}{\hspace{20pt}}

\makeatletter
\newtcbox{\code}{on line, fontupper=\small\ttfamily, boxrule=0.5pt, arc=2pt, coltext=codefg, colback=codebg, colframe=codebg, boxsep=0pt, shrink tight, extrude by=2pt}
\makeatother

\def\checkmark{\tikz\fill[scale=0.4, color=smartdiagram2](0,.35) -- (.25,0) -- (1,.7) -- (.25,.15) -- cycle;} 

%Make \Chaptername return print the name of the chapter
\let\Chaptermark\chaptermark
\def\chaptermark#1{\def\Chaptername{#1}\Chaptermark{#1}}
\let\Sectionmark\sectionmark
\def\sectionmark#1{\def\Sectionname{#1}\Sectionmark{#1}}
\let\Subsectionmark\subsectionmark
\def\subsectionmark#1{\def\Subsectionname{#1}\Subsectionmark{#1}}
\let\Subsubsectionmark\subsubsectionmark
\def\subsubsectionmark#1{\def\Subsubsectionname{#1}\Subsubsectionmark{#1}}

%defT will print T#1 and set a label to T:<ChapterName>:<#1>
%Be aware that this means that any task can only refer to the task within its own chapter
\newcounter{exno}
\crefformat{exno}{#1}
\newcommand{\defT}[1]{\refstepcounter{exno}\label{T:\Chaptername:#1}T#1}

\newcommand{\T}[1]{%
	\hyperref[T:\Chaptername:#1]{T#1}\xspace
}
